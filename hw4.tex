\documentclass{jhwhw}
\usepackage{amsmath}
\usepackage{amssymb}
%\usepackage{tikz}
%\usetikzlibrary{arrows}
%\usepackage[makeroom]{cancel}
%\usetikzlibrary{patterns}
\title{MATH 5510: Topology: HW 4}
\author{Markus Foote}

\newcommand{\R}{{\mathbb R}}
\newcommand{\C}{{\mathbb C}}
\newcommand{\Z}{{\mathbb Z}}
\newcommand{\Q}{{\mathbb Q}}
\newcommand{\N}{{\mathbb N}}
\newcommand{\T}{{\mathcal T}}
\newcommand{\B}{{\mathcal B}}
\begin{document}
\problem{}%1
Let $X$ be a topological space, give $X\times X$ the product topology, and let the \lq\lq diagonal"  $\Delta\subset X\times X$ be defined by 
$$
\Delta =\{ (x,x): x\in X\}.
$$
\begin{enumerate}
	
	\item Prove that $X$ is Hausdorff if and only if $\Delta$ is closed in $X\times X$.
	
	\item Use this fact to give another proof of the fact proved in a previous homework problem:  If $Z$ is another topological space, $f,g:Z\to X$ are continuous, and $X$ is Hausdorff, then
	$$
	E(f,g) = \{x\in Z \ | \ f(x) = g(x)\}
	$$ 
	is closed in $Z$.  Make sure your proof takes at most two lines.
\end{enumerate}
\solution{}
\part{}
Suppose $X$ is Hausdorff. $\Delta^{\mathbf{c}}=\{(x,y)\ | \ x\ne y,\ x,y\in X\}$. $X$ being Hausdorff provides $U,V\ \mathrm{ open }\subset X$ with $x\in U$, $y\in V$. $U\times V$ is thus a basis element with $(x,y) \in U\times V$. Because $U$ and $V$ are disjoint from the Hausdorff condition, they contain no common points, so they contain nothing in $\Delta$: $U\times V \subset \Delta^{\mathbf{c}}\implies \Delta^{\mathbf{c}}$ open $\implies \Delta$ closed in $X\times X$.
\\

Suppose $\Delta$ is closed in $X\times X$. $\forall\ (x,y)\in \Delta^{\mathbf{c}}\  \exists\ U\times V$ basis element with $(x,y)\in U\times V \subset \Delta^{\mathbf{c}}$. $U\times V \subset \Delta^{\mathbf{c}} \implies U\cap V=\emptyset$. Any $x\ne y\in X \implies (x,y)\in \Delta^{\mathbf{c}}$ with a basis element containing $(x,y)$ which provides open, disjoint $U,\ V$ with $x\in U$, $y\in V$, thus $X$ is Hausdorff.


\part{}
\begin{gather}
\Delta = \{(f(x),g(y))\ | \ f(x) = g(y)\} \subset X \times X \text{ s.t. } E(f,g) = F^{-1}(\Delta) \\
X \ \text{Hausdorff}\implies \Delta \ \text{closed.} \ f,g \ \text{cont.} \implies \left( \Delta\ \text{closed} \implies E(f,g)\ \text{closed} \right).
\end{gather}
\\


\problem{} %2
Give an example of a topological space $X$ and a compact subset $C\subset X$ with $C$ not closed in $X$.
\solution{}
Let $X=\left\{ 0,1\right\}$ and give this space the topology $\T=\left\{ \emptyset,\  \{1\},\ X\right\}$. The open subset $C=\left\{ 1\right\}$ is compact because the only open cover $\B =\left\{0,1\right\}$ has the trivial finite subcover of $U_{\alpha_1}=0, U_{\alpha_2}=1$. 


\problem{}%3
Let $X$ be a compact Hausdorff space, and let $A,B\subset X$ be closed sets which are  disjoint : $A\cap B =\emptyset$.  Prove that there are open sets $U,V\subset X$ with $A\subset U$, $B\subset V$, and $U\cap V = \emptyset$. 

\solution{}
%http://math.stackexchange.com/questions/923600/disjoint-compact-subsets-of-a-hausdorff-space-are-separated-by-disjoint-open-nei
Let $b \in B$. For each $a \in A$ we have $a \in U_a ,\ b \in V_a $ with $U_a, V_a$ open such that $U_a \cap V_a = \emptyset$ because $X$ is Hausdorff. The collection $\{U_a\}$ forms an open cover of $A$, so as A is compact there exists a finite subcover of A $U_{a_1}, U_{a_2}, \ldots, U_{a_k}$. Let $U_b = U_{a_1} \cup U_{a_2} \cup \dots \cup U_{a_k}$ which is finite union of open sets, so it is open. Now let $V_b = V_{a_1} \cap V_{a_2} \cap \dots \cap V_{a_k}$. The $\{V_b\}$ form an open cover of $B$, and for each $b \in B, V_b \cap U_b = \emptyset$.  Note that for every $b$, $A \subset U_b$. 
\\

Because $V_b$ forms an open cover of $B$ and B compact, there exists a finite number of them such that $ \bigcup V_{b_i}$ is a finite subcover of $B$. This union of open sets is also open. The union of $\{U_{b_i}\}$ is likewise open. $A$ is contained in each $U_{b_i}$, so their union also contains $A$. We know that each $U_{b_i}$ is disjoint from $V_{b_i}$ so it follows that $\bigcup U_{b_i} \cap\  \bigcap V_{b_i}  = \emptyset$, thus $\bigcup U_{b_i}$ and $\bigcap V_{b_i}$ are the open sets that are disjoint and contain $A\subset\bigcup U_{b_i} ,\ B\subset\bigcap V_{b_i} $.  



\problem{}%4
\begin{enumerate}
	\item Let $(X,\T)$ be a topological space and let $\B$ be a basis for $\T$.  Prove that $(X,\T)$ is compact if and only if every cover of $X$ by elements of $\B$ has a finite sub-cover. 
	\item Let $X$ and $Y$ be compact topological spaces and let $X\times Y$ be their product, with the product topology.   Prove that $X\times Y$ is compact.
	
\end{enumerate}
\solution{}
%http://math.stackexchange.com/questions/718301/prove-that-x-is-compact-if-and-only-if-every-cover-of-x-by-members-of-b-ha
\part{}%a

Suppose that $(X,\T)$ is compact. Using the definition for a basis $\B$ of a topological space that the elements $B_i \in \B$ form a cover of $\T$, which has $X \in \T$, we can form a cover of $X$ from elements $B_i$. Because $\B$ is a basis for $\T$, each element $B_i$ is an open set, which means that the cover of $X$ is open, and because $X$ is compact, the cover has a finite subcovering.
\\

Suppose that every covering of $X$ by $B_i$ has a finite subcover. Let  $U_{\alpha}, \ \alpha \in A$ be an open covering of $X$.  By definition, each element of $U_{\alpha}$ is a union of elements of $\B$. Now take the set $U_{\alpha}$ such that $\B \subset U_{\alpha}$. This $U_{\alpha}$, by definition, is an open covering of $X$ by elements $B_i \in \B$.  Since we assumed that every cover has a finite subcover, this $U_{\alpha}$ has a finite subcovering $B_1,B_2, \dots, B_i$ with each one contained in at least one element of $U_{\alpha}, \alpha \in A$. For each $B_i$, we can choose a $U_i$ such that $B_i \subset U_i$, which means that $\{U_i\}$ is a finite subcovering of $X$, which implies that $X$ is compact.

\part{}%b
%http://math.stackexchange.com/questions/567335/cartesian-product-of-compact-sets-is-compact

Let $\{O_{\alpha}\}_{\alpha \in A}$ be an open cover of $X \times Y$. For each $(x,y) \in X\times Y$ we can choose some $\alpha=\alpha(x,y)$ such that $(x,y) \in O_{\alpha(x,y)}$. Because of how it was constructed, $O_{\alpha(x,y)}$ is open, which means $(x,y)$ is contained in some open rectangle $R \subset O_{\alpha(x,y)}$ where $R = U_{(x,y)} \times V_{(x,y)}$ with $U_{(x,y)} \subset X$ and $V_{(x,y)} \subset Y$.  
\\

Fix $x$, and allow $y$ to vary. For every point $(x,y)$ the point is contained in an open rectangle in the product $X \times Y$, and that rectangle is the product of a subset of $X$ with a subset of $Y$. Choosing several points, we see that the collection of sets $\{V_{(x,y)}\}_{y \in Y}$ is an open cover of $Y$. Because $Y$ is compact, we can find a finite cover $\{V_{(x,y_i(x))}\}$ of $Y$ that consists of finitely many open sets containing points $\{(x,y_i(x))\}$.
\\

Let $U_x = \bigcap_{i} U_{(x,y_i(x))}$. Because $U_x$ is the intersection of finitely many open sets, it is itself open. Using that $X$ is compact, there are finitely many $x_j$ such that $\{U_{x_j}\}$ forms an open cover of $X$. It follows that $\{O_{x_i,y_i(x)}\}$ for any $i,j$ combination is a finite cover of $X \times Y$, which means that $X \times Y$ is compact.

\problem{}%5
We have seen that the Cantor set can be described as the set $\{0,2\}^\N$ of infinite sequences of zeros and twos, which is in bijective correspondence with the more convenient set $\{0,1\}^\N$ of infinite sequences of zeros and ones.  This choice has  the advantage that $\{0,1\}$ can be  naturally identified with $\Z/2$, the integers modulo two, which forms a group under addition:  $ 0 + 0 = 1+ 1 = 0, 0+1 = 1+ 0 = 1$.  In this way the Cantor set becomes a group, by pointwise addition of sequences: $\{a_i\} + \{b_i\} = \{a_i + b_i \}$.  

\emph{ Prove that this operation is continuous}. This means that the  addition  map
$$
A: \{0,1\}^\N\times\{0,1\}^\N \to \{0,1\}^\N 
$$
defined by
$$
A(\{a_1,a_2,\dots, a_i,\dots  \},\{b_1,b_2,\dots, b_i,\dots\})  = \{a_1 + b_1, a_2+b_2,\dots , a_i +b_i,\dots \}
$$
is continuous, where $\{0,1\}^\N$ is given the (infinite) product topology, and $\{0,1\}^\N\times\{0,1\}^\N$ the product (two factors) of the infinite product topologies in each factor.

\emph{Suggestion}:  Fix $i_0\in\N$, and  fix an open set $U\subset \{0,1\}$ (so  $U $ is one of $\emptyset, \{0\}, \{1\},\{0,1\}$).  The sets $A^{-1}(\{ \{a_i\} \ | \ a_{i_0}\in U\})$ form a sub-basis (see notes, v1,  3.4.3)  for the topology of $\{0,1\}^\N$, so it is enough to show that $A^{-1}(\{ \{a_i\} \ | \ a_{i_0}\in U\})$ is open.

\solution{}
Fix $i_0 \in \N$. Let $U \subset \{0,1\}$ be open. We show that the preimage, $A^{-1}(\{ \{a_i\} \ | \ a_{i_0}\in U\})$, is open for each possible open set $U$:
\\

For $a_{i_0}\in \emptyset$:
\begin{equation}
\text{Trivially, } A^{-1}(\emptyset) = \emptyset \text{ which is open.}
\end{equation}

For $a_{i_0}\in \{0\}$: 
\begin{align}A^{-1}(\{0\}) &= (\{0\},\{0\})\cup(\{1\},\{1\}) \\
&= (\{0,1\},\{0,1\}) \text{ which is open.}
\end{align}

For $a_{i_0}\in \{1\}$: 
\begin{align}A^{-1}(\{1\}) &= (\{0\},\{1\})\cup(\{1\},\{0\}) \\
&= (\{0,1\},\{1,0\})\\
&= (\{0,1\},\{0,1\}) \text{ which is open.}
\end{align}

For $a_{i_0}\in \{0,1\}$: 
\begin{align}A^{-1}(\{0,1\}) &= (\{0,1\},\{0,1\})\cup(\{0,1\},\{0,1\}) \\
&= (\{0,1\},\{0,1\}) \text{ which is open.}
\end{align}

This shows the preimage of all open sets remains open, thus $A$ is continuous. 
\\

Note that the value at the $i_0$ position depends {\em only} on the corresponding values at the $i_0$ position in the product space \textendash position gets preserved.
\end{document}