% !TeX encoding = UTF-8
\documentclass{jhwhw}
\usepackage{amsmath}
\usepackage{amssymb}
%\usepackage{indentfirst}
%\usepackage{tikz}
%\usetikzlibrary{arrows}
%\usepackage[makeroom]{cancel}
%\usetikzlibrary{patterns}
\title{MATH 5510: Topology: HW 6}
\author{Markus Foote}

\newcommand{\R}{{\mathbb R}}
\newcommand{\C}{{\mathbb C}}
\newcommand{\Z}{{\mathbb Z}}
\newcommand{\Q}{{\mathbb Q}}
\newcommand{\N}{{\mathbb N}}
\newcommand{\T}{{\mathcal T}}
\newcommand{\B}{{\mathcal B}}
\begin{document}
\problem{}%1
A topological space $X$ is said to be \emph{locally connected} if it has a basis consisting of connected open sets.
\begin{enumerate}
	\item Prove that if $X$ is locally connected, then its connected components are open.
	\item Prove that if $X$ is locally connected and $Y$ is the space of its connected components, with the identification topology, then $Y$ is discrete.
\end{enumerate}
\solution{}
\part{}%a
Let $x\in C$, where $C$ is a connected component of $X$. By definition, $x$ is contained in some open connected subset $U$ of $X$. Since $C$ is a maximal connected set containing $x$, $x\in U\subseteq C$. Because $C$ is thus the union of all possible $U$ and openness is closed under arbitrary union, this shows that $C$ is open in $X$.
\part{}%b
In the identification topology on Y, the open sets are defined as sets of which the pre-image is open. If Y is to be discrete, then every set must be open. Because openness is closed under arbitrary union, if we show that all points are open then every set will be open. A point in the identification topology will be open if and only if the pre-image of that point is open. The pre-image of a point is a connected component. All connected components are open by (a). Thus points in Y are open, thus Y is discrete.

\problem{} %2
Recall that a topological space $X$ is said to be \emph{totally disconnected} if for all $x\in X$, the connected component $C_x$ of $X$ containing $x$ is simply $\{x\}$.  %A discrete space is totally disconnected.  The purpose of this exercise is to show that the Cantor set $\{0,1\}^\N$ is totally disconnected, but is the opposite of discrete:  every point is an accumulation point.  Here, as usual, the two point space $\{0,1\}$ has the discrete topology.
\begin{enumerate}
	
	\item Let $X$ be a topological space, and suppose that for all $x,y \in X, x\ne y$, there exists a continuous  function $f:X\to \{0,1\}$ with $f(x)\ne f(y)$. Prove that $X$ is totally disconnected.
	
	
	\item Prove that $\{0,1\}^\N$ (with the product topology) is totally disconnected.
	
	\item Let $ x \in \{0,1\}^\N$ be an arbitrary point.  Prove that $x$ is an accumulation point, that is, given any nbd $U $ of $x$ there exists $y\in U$ such that $y\ne x$.  
\end{enumerate}
\solution{}
\part{}%a
Because a connected subset has a continuous, constant function that maps that subset to $\{0,1\}$ (Thm 5.2), we know that there are no connected subsets that contain both $x$ and $y$. (If there were such a connected subset, then a continuous constant function must exist for that set, which would violate $f(x)\neq f(y)$). Thus connected subsets cannot contain more than one point $\implies$ Connected subsets of X are single points $\implies$ X is totally disconnected.
\part{}%b
Here we apply (a) because the product topology gives continuous projection functions which, in this case, map to $\{0,1\}$. For any $x,y\in \{0,1\}^\N$, pick the projection to a factor that is different between the two points. This function satisfies requirements for (a), thus $\{0,1\}$ is totally disconnected.
\part{}%c
A neighborhood $U$ is the set of points that are equal up until some factor $n$, and differ in at least one factor after the first $n$ factors. Then $y$ can be the point that is equal to x for the first $n$ factors and differ in one or more factors after $n$. Thus $x$ is an accumulation point.

\problem{}%3
Let $S^2\subset \mathbb{R}^3$ be the unit sphere centered at the origin, and  let $N = (0,0,1)$ be the north pole.  \emph{Stereographic  projection from the north pole} is the map   $f:S^2\setminus\{N\}\to\mathbb{R}^2$ defined by letting  $f(p)$ be the point of intersection with $\mathbb{R}^2 = \{(x,y,z)\in\mathbb{R}^3:z=0\}$ of the straight line through $N$ and $p$.  
\begin{enumerate}
	\item Find a formula for $f$
	\item Find a formula for the inverse map $g:\mathbb{R}^2\to S^2\setminus 
	\{N\}$. 
	\item
	Use stereographic projections from both the north and south poles to cover $S^2$ by the domain of two coordinate charts to $\R^2$ with a smooth transition function.  Conclude that $S^2$ is a smooth surface.
\end{enumerate} 

\solution{}

\part{}%a
Based on radially invariant similar triangles,
\begin{equation}
f_N(x,y,z) = \left(\frac{\sqrt{x^2+y^2}}{1-z}\frac{y}{x^2+y^2} , \frac{\sqrt{x^2+y^2}}{1-z}\frac{x}{x^2+y^2}\right)
\end{equation}

\part{}%b
\begin{equation}
g_N(x,y) = \left(\sqrt{1-\left(\frac{x^2+y^2-1}{x^2+y^2+1}\right)^2}\frac{y}{x^2+y^2},\sqrt{1-\left(\frac{x^2+y^2-1}{x^2+y^2+1}\right)^2}\frac{x}{x^2+y^2},\frac{x^2+y^2-1}{x^2+y^2+1}\right)
\end{equation}
\part{}%c
\begin{gather}
f_S(x,y,z) = \left(\frac{\sqrt{x^2+y^2}}{1+z}\frac{y}{x^2+y^2} , \frac{\sqrt{x^2+y^2}}{1+z}\frac{x}{x^2+y^2}\right)\\
g_S(x,y) = \left(\sqrt{1-\left(\frac{x^2+y^2-1}{x^2+y^2+1}\right)^2}\frac{y}{x^2+y^2},\sqrt{1-\left(\frac{x^2+y^2-1}{x^2+y^2+1}\right)^2}\frac{x}{x^2+y^2},-\frac{x^2+y^2-1}{x^2+y^2+1}\right)
\end{gather}
The compositions of these functions $f_N\circ g_S$ and $f_S\circ g_N$ are the transition functions. These functions are infinitely differentiable so the surface is smooth.
\end{document}